\documentclass[11pt]{article}
\title { implicit search}
\begin{document}
\maketitle
%\section{}
%\subsection{}
Results of context search
\newline  [  1  ]   We  further showed that adaptation to gradual perturbations could enable savings,  which was supported by enhanced   \textbf {  implicit  }   learning.
\newline  [  2  ]   Our study provides supporting  evidence that long-term retention of motor adaptation is possible without  forming or recalling a cognitive strategy, and the interplay between   \textbf {  implicit  }    and explicit learning critically depends on the specifics of learning protocol  and available sensory feedback.NEW & NOTEWORTHY Savings in motor learning  sometimes refers to faster learning when one encounters the same perturbation  again.
\newline  [  3  ]  Epub 2020 Feb 6.  Task Errors Drive Memories That Improve Sensorimotor Adaptation.  Leow LA(1), Marinovic W(2), de Rugy A(3), Carroll TJ(4).  Author information: (1)Centre for Sensorimotor Performance, School of Human Movement and Nutrition  Sciences, University of Queensland, Brisbane, 4072 Queensland, Australia,  l.leow@uq.edu.au. (2)School of Psychology, Curtin University, Bentley, 6102 Western Australia,  Australia, and. (3)Institut de Neurosciences Cognitives et Intégratives d'Aquitaine, Centre  National de la Recherche Scientifique Unité Mixte de Recherche 5287, Université  de Bordeaux, France 33076. (4)Centre for Sensorimotor Performance, School of Human Movement and Nutrition  Sciences, University of Queensland, Brisbane, 4072 Queensland, Australia.  Traditional views of sensorimotor adaptation (i.e., adaptation of movements to  perturbed sensory feedback) emphasize the role of automatic,   \textbf {  implicit  }   correction  of sensory prediction errors.
\newline  [  4  ]   SANS individuals showed larger, more persistent  after-effects, suggesting a shift from relying on cognitive to more   \textbf {  implicit  }    processing of adaptive behaviors.
\newline  [  5  ]   Individuals who developed signs  of SANS seem to rely more on   \textbf {  implicit  }   rather than cognitive processing of  adaptive behaviors than subjects who did not present signs of SANS.  DOI: 10.1152/jn.00306.2020
\newline  [  6  ]   We then apply the  derived generalization function to our model and successfully simulate and fit  the time course of   \textbf {  implicit  }   adaptation across three behavioral experiments.
\newline  [  7  ]   A new method to isolate   \textbf {  implicit  }   learning  involves the use of a "clamped" visual perturbation in which, during a reaching  movement, visual feedback is limited to a cursor that follows an invariant  trajectory offset from the target by a fixed angle.
\newline  [  8  ]   Several such approaches have recently  emerged which, taken together, suggest that two fundamental systems operate  together to achieve the adapted state: one system learns slowly, is   \textbf {  implicit  }  , is  temporally stable over short breaks, is expressible at low reaction times, and  its properties do not change based on experience.
\newline  [  9  ]   J Neurosci. 2020 Mar 18;40(12):2498-2509. doi: 10.1523/JNEUROSCI.1862-19.2020.  Epub 2020 Feb 7.  Spatially Distinct Beta-Band Activities Reflect Implicit Sensorimotor Adaptation  and Explicit Re-aiming Strategy.  Jahani A(1), Schwey A(1), Bernier PM(2)(3), Malfait N(4).  Author information: (1)Institut de Neurosciences de la Timone, Unité Mixte de Recherche 7289, Centre  National de la Recherche Scientifique/Aix Marseille Université, 13385 Marseille  Cedex 5, France. (2)Département de Kinanthropologie, Faculté des Sciences de l'activité Physique,  Université de Sherbrooke, Sherbrooke, Québec J1H 5N4, Canada, and. (3)Centre de Recherche du CHUS, Université de Sherbrooke, Sherbrooke, Québec J1H  5N4, Canada. (4)Institut de Neurosciences de la Timone, Unité Mixte de Recherche 7289, Centre  National de la Recherche Scientifique/Aix Marseille Université, 13385 Marseille  Cedex 5, France, nicole.malfait@univ-amu.fr nmalfait@gmail.com.  Previous research suggests that so-called   \textbf {  implicit  }   and explicit processes of  motor adaptation are implemented by distinct neural structures.
\newline  [  10  ]   Here we tested  whether   \textbf {  implicit  }   sensorimotor adaptation and strategic re-aiming used to reduce  movement error are reflected by spatially distinct EEG oscillatory components.  We analyzed beta-band oscillations (∼13-30 Hz), which have long been linked to  sensorimotor functions, at the time when these adaptive processes intervene for  movement planning.
\newline  [  11  ]   We hypothesized that beta-band activity within sensorimotor  regions relates to   \textbf {  implicit  }   adaptive processes, whereas beta-band activity  within medial motor areas reflects deliberate re-aiming.
\newline  [  12  ]   We propose that the reduction  in lateral central beta power reflects an increased weighting of peripheral  sensory information   \textbf {  implicit  }  ly triggered when an adaptive change in the  sensorimotor mapping is required, whereas the reduction in medial frontal  beta-band activity relates to the inhibition of automatic motor responses in  favor of cognitively controlled movements.SIGNIFICANCE STATEMENT Behavioral and  modeling studies have proposed that so-called implicit and explicit components  of motor adaptation recruit different neural circuits.
\newline  [  13  ]   Analyzing EEG signals at the time they influence movement planning,  during the foreperiod, we found that beta power within lateral central regions  was decreased when a change in visual conditions required   \textbf {  implicit  }   sensorimotor  remapping, which may reflect enhanced sensory processing when internal-model  predictions are rendered uncertain.
\newline  [  14  ]   Research has focused on how this form of error-based learning takes  place in an   \textbf {  implicit  }   and automatic manner.
\newline  [  15  ]   Epub 2020  Nov 25.  The effect of visual uncertainty on   \textbf {  implicit  }   motor adaptation.  Tsay JS(1)(2), Avraham G(1)(2), Kim HE(3), Parvin DE(1)(2), Wang Z(1), Ivry  RB(1)(2).  Author information: (1)Department of Psychology, University of California, Berkeley, California. (2)Helen Wills Neuroscience Institute, University of California, Berkeley,  California. (3)Department of Physical Therapy, University of Delaware, Newark, Delaware.  Sensorimotor adaptation is influenced by both the size and variance of error  information.
\newline  [  16  ]   Here, we used a novel visuomotor task that  isolates the contribution of   \textbf {  implicit  }   adaptation, independent of error size.
\newline  [  17  ]   A naïve,   \textbf {  implicit  }   group and a group of  subjects using explicit strategies adapted to 20°, 40° and 60° cursor rotations  in different adaptation blocks that were each followed by determination of  awareness and unawareness indices.
\newline  [  18  ]  9.  Task errors contribute to   \textbf {  implicit  }   aftereffects in sensorimotor adaptation.  Leow LA(1), Marinovic W(2), de Rugy A(3), Carroll TJ(1).  Author information: (1)Centre for Sensorimotor Performance, School of Human Movement and Nutrition  Sciences, The University of Queensland, Brisbane, Australia. (2)School of Psychology, Curtin University, Perth, Australia. (3)Institut de Neurosciences Cognitives et Intégratives d'Aquitaine, CNRS, UMR  5287, Université de Bordeaux, St Lucia, France.  Perturbations of sensory feedback evoke sensory prediction errors (discrepancies  between predicted and actual sensory outcomes of movements), and reward  prediction errors (discrepancies between predicted rewards and actual rewards).  When our task is to hit a target, we expect to succeed in hitting the target,  and so we experience a reward prediction error if the perturbation causes us to  miss it.
\newline  [  19  ]   Hence, task errors contribute to   \textbf {  implicit  }   adaptation resulting from  sensory prediction errors.
\newline  [  20  ]  eCollection 2017.  Cerebellar anodal tDCS increases   \textbf {  implicit  }   learning when strategic re-aiming is  suppressed in sensorimotor adaptation.  Leow LA(1), Marinovic W(1)(2), Riek S(1), Carroll TJ(1).  Author information: (1)Centre for Sensorimotor Performance, School of Human Movement and Nutrition  Sciences, Building 26B, The University of Queensland, Brisbane, Australia. (2)School of Psychology and Speech Pathology, Building 401, Curtin University,  Bentley, Perth, WA, Australia.  Neurophysiological and neuroimaging work suggests that the cerebellum is  critically involved in sensorimotor adaptation.
\newline  [  21  ]   It is known, however, that behavioural responses to  sensorimotor perturbations reflect both explicit processes (such as volitional  aiming to one side of a target to counteract a rotation of visual feedback) and    \textbf {  implicit  }  , error-driven updating of sensorimotor maps.
\newline  [  22  ]   The contribution of the  cerebellum to these explicit and   \textbf {  implicit  }   processes remains unclear.
\newline  [  23  ]   After exposure to the rotation, we evaluated   \textbf {  implicit  }    remapping in no-feedback trials after providing participants with explicit  knowledge that the rotation had been removed.
\newline  [  24  ]   Thus, cerebellar  anodal tDCS increased   \textbf {  implicit  }   remapping during sensorimotor adaptation,  irrespective of preparation time constraints.
\newline  [  25  ]   However, across three experiments with human participants  (N = 47, 26 female), we show that these mechanisms can be dissociated based on  the properties of   \textbf {  implicit  }   adaptation under mirror-reversed visual feedback.  Although mirror reversal is an extreme perturbation, it still elicits implicit  adaptation; however, this adaptation acts to amplify rather than to reduce  errors.
\newline  [  26  ]   We show that the pattern of this adaptation over time and across targets  is consistent with direct policy learning but not forward-model-based learning.  Our findings suggest that the forward-model-based theory of adaptation needs to  be re-examined and that direct policy learning provides a more plausible  explanation of   \textbf {  implicit  }   adaptation.SIGNIFICANCE STATEMENT The ability of our  brain to adapt movements in response to error is one of the most widely studied  phenomena in motor learning.
\newline  [  27  ]   This deficit has been attributed to impairment in sensory  prediction error-based updating of an internal forward model, a form of   \textbf {  implicit  }    learning.
\newline  [  28  ]   This successful use of an  instructed aiming strategy presents a paradox: In adaptation tasks, why do  individuals with cerebellar damage not come up with an aiming solution on their  own to compensate for their   \textbf {  implicit  }   learning deficit? To explore this question,  we employed a variant of a visuomotor rotation task in which, before executing a  movement on each trial, the participants verbally reported their intended aiming  location.
\newline  [  29  ]   Compared with healthy control participants, participants with  spinocerebellar ataxia displayed impairments in both   \textbf {  implicit  }   learning and  aiming.
\newline  [  30  ]   This dual deficit does not appear to  be related to the increased movement variance associated with ataxia: Healthy  undergraduates showed little change in   \textbf {  implicit  }   learning or aiming when their  movement feedback was artificially manipulated to produce similar levels of  variability (experiment 3).
\newline  [  31  ]   Here, we investigated the interaction  of these two forms of motor learning by having subjects adapt to predictable  forces imposed by a robotic manipulandum while simultaneously reaching to an    \textbf {  implicit  }   sequence of targets.
\newline  [  32  ]   In  response to changes in visual feedback of movements, explicit (cognitive) and    \textbf {  implicit  }   (automatic) learning processes adapt movements to counter errors.  However, movements rarely occur in isolation.
\newline  [  33  ]   The extent to which explicit and    \textbf {  implicit  }   processes drive sensorimotor adaptation when multiple movements occur  simultaneously, as in the real world, remains unclear.
\newline  [  34  ]   However, the aftereffect, an indicator of    \textbf {  implicit  }   motor adaptation, was attenuated with delayed error feedback,  consistent with the hypothesis that a different learning process supports  performance under delay.
\newline  [  35  ]   We tested this by employing a task that dissociates the  contribution of explicit strategies and   \textbf {  implicit  }   adaptation.
\newline  [  36  ]   We find that  explicit aiming strategies contribute to the majority of the learning curve,  regardless of delay; however,   \textbf {  implicit  }   learning, measured over the course of  learning and by aftereffects, was significantly attenuated with delayed  error-based feedback.
\newline  [  37  ]   We  observe no evidence of savings in   \textbf {  implicit  }   error-based adaptation.  Copyright © 2015 the authors 0270-6474/15/3514386-11$15.00/0.  DOI: 10.1523/JNEUROSCI.1046-15.2015 PMCID: PMC4683692
\newline  [  38  ]   When human participants encounter  repeated or consistent perturbations, their corrections for the experienced  errors are larger compared with when the perturbations are new or inconsistent.  Such modulations of error sensitivity were traditionally considered to be an    \textbf {  implicit  }   process that does not require attentional resources.
\newline  [  39  ]   In recent years,  the   \textbf {  implicit  }   view of motor adaptation has been challenged by evidence showing a  contribution of explicit strategies to learning.
\newline  [  40  ]   In addition, delaying the visual feedback, a  manipulation that affects   \textbf {  implicit  }   learning, did not influence error sensitivity  under consistent perturbations.
\newline  [  41  ]   These results suggest that increases of learning  rate in consistent environments are attributable to an explicit rather than    \textbf {  implicit  }   process in sensorimotor adaptation.NEW & NOTEWORTHY The consistency of  an external perturbation modulates error sensitivity and the motor response.
\newline  [  42  ]   The  roles of explicit and   \textbf {  implicit  }   processes in this modulation are unknown.
\newline  [  43  ]   When the   \textbf {  implicit  }   system  is manipulated by delaying feedback, sensitivity to a consistent perturbation  does not change.
\newline  [  44  ]  ahead of print.  Gait symmetric adaptation: Comparing effects of   \textbf {  implicit  }   visual distortion  versus split-belt treadmill on aftereffects of adapted step length symmetry.  Chunduru P(1), Kim SJ(2), Lee H(3).  Author information: (1)School of Biological and Health Systems Engineering, Arizona State  University, Tempe, AZ 85287, United States. (2)Biomedical Engineering, California Baptist University, Riverside, CA 92504,  United States.
\newline  [  45  ]   We have previously shown that subjects can adapt spatial aspects of  walking to an   \textbf {  implicit  }  ly imposed distortion of visual feedback of step length.  To further investigate the storage benefit of an implicit process engaged in  visual feedback distortion, we compared the robustness of aftereffects acquired  by visual feedback distortion, versus split-belt treadmill walking.
\newline  [  46  ]   For the  visual distortion trial, we   \textbf {  implicit  }  ly distorted the visual representation of  subjects' gait symmetry, whereas for the split-belt trial, the speed ratio of  the two belts was gradually adjusted without visual feedback.
\newline  [  47  ]   Visual distortion adaptation may involve the interaction  and integration of the change in motor strategy and   \textbf {  implicit  }   process in  sensorimotor adaptation.
\newline  [  48  ]   Although it should be clarified more clearly through  further studies, the findings of this study suggest that gait control employs  distinct adaptive processes during the visual distortion and split-belt walking  and also the level of reliance of an   \textbf {  implicit  }   process may be greater in the  visual distortion adaptation than the split-belt walking adaptation.  Copyright © 2019 Elsevier B.V.
\newline  [  49  ]  Nov 2.  Individual differences in   \textbf {  implicit  }   motor learning: task specificity in  sensorimotor adaptation and sequence learning.  Stark-Inbar A(1), Raza M(2), Taylor JA(3), Ivry RB(2)(4).  Author information: (1)Department of Psychology, University of California, Berkeley, California;  alit.stark@gmail.com. (2)Department of Psychology, University of California, Berkeley, California. (3)Department of Psychology, Princeton University, Princeton, New Jersey. (4)Helen Wills Neuroscience Institute, University of California, Berkeley,  California; and.  In standard taxonomies, motor skills are typically treated as representative of  implicit or procedural memory.
\newline  [  50  ]   We examined two emblematic tasks of   \textbf {  implicit  }    motor learning, sensorimotor adaptation and sequence learning, asking whether  individual differences in learning are correlated between these tasks, as well  as how individual differences within each task are related to different  performance variables.
\newline  [  51  ]   For the latter, greater learning was associated  with higher variability and slower reaction times, factors that may facilitate  the spread of activation required to form predictive, sequential associations.  Interestingly, learning measures of the different tasks were not correlated.  Together, these results oppose a shared process for   \textbf {  implicit  }   learning in  sensorimotor adaptation and sequence learning and provide insight into the  factors that account for individual differences in learning within each task  domain.
\newline  [  52  ]   NEW & NOTEWORTHY: We investigated individual differences in the ability to    \textbf {  implicit  }  ly learn motor skills.
\newline  [  53  ]   We found that two commonly used  tasks of   \textbf {  implicit  }   learning, visuomotor adaptation and the alternating serial  reaction time task, exhibited good test-retest reliability in measures of  learning and performance.
\newline  [  54  ]   However, the learning measures did not correlate  between the two tasks, arguing against a shared process for   \textbf {  implicit  }   motor  learning.  Copyright © 2017 the American Physiological Society.  DOI: 10.1152/jn.01141.2015 PMCID: PMC5253399
\newline  [  55  ]  22.  Martial arts training is related to   \textbf {  implicit  }   intermanual transfer of visuomotor  adaptation.  Werner S(1)(2), Hasegawa K(3), Kanosue K(3), Strüder HK(2), Göb T(2), Vogt  T(1)(3).  Author information: (1)Institute of Professional Sport Education and Sport Qualifications, German  Sport University, Cologne, Germany. (2)Institute of Movement and Neurosciences, German Sport University, Cologne,  Germany. (3)Faculty of Sport Sciences, Waseda University, Tokorozawa, Japan.  Recent work identified an explicit and implicit transfer of sensorimotor  adaptation with one limb to the other, untrained limb.
\newline  [  56  ]   Here, we pursue the idea  that different individual factors contribute differently to the amount of  explicit and   \textbf {  implicit  }   intermanual transfer.
\newline  [  57  ]   We measured the total  amount of intermanual transfer (explicit plus   \textbf {  implicit  }  ) by telling the  participants to repeat what was learned during adaptation, and the amount of  implicit transfer by instructing the participants to refrain from using what was  learned and to perform movements as during baseline instead.
\newline  [  58  ]   We found no  difference between the total intermanual transfer of judokas and running  experts, with mean absolute transfer values of 42.4° and 47.0°. Implicit  intermanual transfer was very limited, but larger in judokas than in general  sports athletes, with mean values of 5.2° and 1.6°. A multiple linear regression  analysis further revealed that total intermanual transfer, which mainly  represents the explicit transfer, is related to awareness of the perturbation,  while   \textbf {  implicit  }   intermanual transfer can be predicted by judo training, amount of  total training, speed of adaptation, and handedness scores.
\newline  [  59  ]   Explicit and   \textbf {  implicit  }   contributions to learning in a sensorimotor adaptation  task.  Taylor JA(1), Krakauer JW, Ivry RB.  Author information: (1)Department of Psychology, Princeton University, Princeton, New Jersey 08544,  Department of Neurology and Department of Neuroscience, Johns Hopkins  University, Baltimore, Maryland 21218, and Department of Psychology and Helen  Wills Neuroscience Institute, University of California, Berkeley, California  94720.  Visuomotor adaptation has been thought to be an implicit process that results  when a sensory-prediction error signal is used to update a forward model.
\newline  [  60  ]   We had two main hypotheses: the  contribution of explicit learning would be modulated by instruction and the  contribution of   \textbf {  implicit  }   learning would be modulated by the form of error  feedback.
\newline  [  61  ]   By directly assaying aiming direction, we could identify the time  course of the explicit component and, via subtraction, isolate the   \textbf {  implicit  }    component of learning.
\newline  [  62  ]   There were marked differences in the time courses of  explicit and   \textbf {  implicit  }   contributions to learning.
\newline  [  63  ]   In contrast,   \textbf {  implicit  }    learning, driven by a sensory-prediction error, was slow and monotonic.  Continuous error feedback reduced the amplitude of explicit learning and  increased the contribution of implicit learning.
\newline  [  64  ]   The presence of instruction  slightly increased the rate of initial learning and only had a subtle effect on    \textbf {  implicit  }   learning.
\newline  [  65  ]   We conclude that visuomotor adaptation, even in the absence  of instruction, results from the interplay between explicit learning driven by  target error and   \textbf {  implicit  }   learning of a forward model driven by prediction  error.  DOI: 10.1523/JNEUROSCI.3619-13.2014 PMCID: PMC3931506
\newline  [  66  ]   The influence of task outcome on   \textbf {  implicit  }   motor learning.  Kim HE(1)(2)(3)(4), Parvin DE(1)(2), Ivry RB(1)(2).  Author information: (1)Department of Psychology, University of California, Berkeley, Berkeley,  United States. (2)Helen Wills Neuroscience Institute, University of California, Berkeley,  Berkeley, United States. (3)Department of Physical Therapy, University of Delaware, Newark, United  States. (4)Department of Psychological and Brain Sciences, University of Delaware,  Newark, United States.  Recent studies have demonstrated that task success signals can modulate learning  during sensorimotor adaptation tasks, primarily through engaging explicit  processes.
\newline  [  67  ]   Here, we examine the influence of task outcome on   \textbf {  implicit  }    adaptation, using a reaching task in which adaptation is induced by feedback  that is not contingent on actual performance.
\newline  [  68  ]   In  this way, the cursor either hit or missed the target, with the former producing  a marked attenuation of   \textbf {  implicit  }   motor learning.
\newline  [  69  ]   We explored different  computational architectures that might account for how task outcome information  interacts with   \textbf {  implicit  }   adaptation.
\newline  [  70  ]   Rather, task outcome may serve as a gain on   \textbf {  implicit  }   adaptation or  provide a distinct error signal for a second, independent implicit learning  process.
\newline  [  71  ]  Mar 30.  A spatial explicit strategy reduces error but interferes with sensorimotor  adaptation.  Benson BL(1), Anguera JA, Seidler RD.  Author information: (1)School of Kinesiology, University of Michigan, Ann Arbor, Michigan, USA.  Although sensorimotor adaptation is typically thought of as an   \textbf {  implicit  }   form of  learning, it has been shown that participants who gain explicit awareness of the  nature of the perturbation during adaptation exhibit more learning than those  who do not.
\newline  [  72  ]   Late in  adaptation, performance was indistinguishable between the explicit and   \textbf {  implicit  }    groups, but the mechanisms underlying performance improvements remained  fundamentally different, as revealed by catch trials.
\newline  [  73  ]   The progression of    \textbf {  implicit  }   recalibration in the explicit group was modulated by the use of an  explicit strategy: these participants showed a lower level of recalibration as  well as decreased aftereffects.
\newline  [  74  ]   This phenomenon may be due to the reduced  magnitude of errors made to the target during adaptation or inhibition of    \textbf {  implicit  }   learning mechanisms by explicit processing.  DOI: 10.1152/jn.00002.2011 PMCID: PMC3118744
\newline  [  75  ]   More importantly, we found that this association varies  with age, task type, and engagement of   \textbf {  implicit  }   versus explicit learning  processes.  Copyright © 2016 Elsevier Ltd.
\newline  [  76  ]   Individual differences in proprioception predict the extent of   \textbf {  implicit  }    sensorimotor adaptation.  Tsay JS(1), Kim HE(2), Parvin DE(1), Stover AR(3), Ivry RB(1).  Author information: (1)Department of Psychology, University of California, Berkeley, United States. (2)Department of Physical Therapy, University of Delaware, United States. (3)Psychology, University of California, Berkeley, United States.  Recent studies have revealed an upper bound in motor adaptation, beyond which  other learning systems may be recruited.
\newline  [  77  ]   While these results do not favor one hypothesis over  the other, they underscore the critical role of proprioception in sensorimotor  adaptation, and moreover, motivate a novel perspective on how these  proprioceptive constraints drive   \textbf {  implicit  }   changes in motor behavior.  DOI: 10.1152/jn.00585.2020
\newline  [  78  ]  print.  An   \textbf {  implicit  }   memory of errors limits human sensorimotor adaptation.  Albert ST(1), Jang J(2), Sheahan HR(3), Teunissen L(4), Vandevoorde K(5)(6),  Herzfeld DJ(2)(7), Shadmehr R(2).  Author information: (1)Department of Biomedical Engineering, Johns Hopkins School of Medicine,  Baltimore, MD, USA. salbert8@jhmi.edu. (2)Department of Biomedical Engineering, Johns Hopkins School of Medicine,  Baltimore, MD, USA. (3)Department of Experimental Psychology, University of Oxford, Oxford, UK. (4)Donders Institute for Brain, Cognition and Behaviour, Radboud University,  Nijmegen, the Netherlands. (5)Leuven Brain Institute, KU Leuven, Leuven, Belgium. (6)Movement Control and Neuroplasticity Research Group, Department of Movement  Sciences, KU Leuven, Leuven, Belgium. (7)Department of Neurobiology, Duke University School of Medicine, Durham, NC,  USA.  During extended motor adaptation, learning appears to saturate despite  persistence of residual errors.
\newline  [  79  ]   These changes in total adaptation could relate to either   \textbf {  implicit  }   or  explicit learning systems.
\newline  [  80  ]   In contrast, when learning depended entirely, or in  part, on   \textbf {  implicit  }   learning, residual errors reappeared.
\newline  [  81  ]   Total   \textbf {  implicit  }    adaptation decreased in the high-variance environment due to changes in error  sensitivity, not in forgetting.
\newline  [  82  ]   These observations suggest a model in which the    \textbf {  implicit  }   system becomes more sensitive to errors when they occur in a consistent  direction.
\newline  [  83  ]   Thus, residual errors in motor adaptation are at least in part caused  by an   \textbf {  implicit  }   learning system that modulates its error sensitivity in response  to the consistency of past errors.  DOI: 10.1038/s41562-020-01036-x
\newline  [  84  ]   Here we evaluated the    \textbf {  implicit  }   and explicit motor learning capabilities of children with DCD.
\newline  [  85  ]   Here we identify constraints that influence this process,  allowing us to explore models of the interaction between strategic and   \textbf {  implicit  }    changes during visuomotor adaptation.
\newline  [  86  ]   Moreover, when we removed visual markers that provided external  landmarks to support a strategy, the degree of drift was sharply attenuated.  These effects are accounted for by a setpoint state-space model in which a  strategy is flexibly adjusted to offset performance errors arising from the    \textbf {  implicit  }   adaptation of an internal model.
\newline  [  87  ]   This incomplete asymptote has been  explained as a consequence of obligatory computations within the   \textbf {  implicit  }    adaptation system, such as an equilibrium between learning and forgetting.
\newline  [  88  ]   A  body of recent work has shown that in standard adaptation tasks, cognitive  strategies operate alongside   \textbf {  implicit  }   learning.
\newline  [  89  ]  2018 Aug 8.  A locomotor learning paradigm using distorted visual feedback elicits strategic  learning.  French MA(1)(2), Morton SM(1)(2), Charalambous CC(1)(3), Reisman DS(1)(2).  Author information: (1)Department of Physical Therapy, University of Delaware , Newark, Delaware. (2)Biomechanics and Movement Science Program, University of Delaware , Newark,  Delaware. (3)Department of Neurology, New York University Langone Health , New York, New  York.  Distorted visual feedback (DVF) during locomotion has been suggested to result  in the development of a new walking pattern in healthy individuals through    \textbf {  implicit  }   learning processes.
\newline  [  90  ]   Recent work in upper extremity visuomotor rotation  paradigms suggest that these paradigms involve   \textbf {  implicit  }   and explicit learning.  Additionally, in upper extremity visuomotor paradigms, the verbal cues provided  appear to impact how a behavior is learned and when this learned behavior is  used.
\newline  [  91  ]   Here, in two experiments in neurologically intact individuals, we tested  how verbal instruction impacts learning a new locomotor pattern on a treadmill  through DVF, the transfer of that pattern to overground walking, and what types  of learning occur (i.e.,   \textbf {  implicit  }   vs. explicit learning).
\newline  [  92  ]   Additionally, the aftereffects observed were significantly  different from the baseline walking pattern, but smaller than the behavior  changes observed during learning, which is uncharacteristic of   \textbf {  implicit  }    sensorimotor adaptation.
\newline  [  93  ]   Based on these  results, we conclude that DVF during locomotion results in a large portion of  explicit learning and a small portion of   \textbf {  implicit  }   learning.
\newline  [  94  ]   NEW & NOTEWORTHY The  results of this study suggest that distorted visual feedback during locomotor  learning involves the development of an explicit strategy with only a small  component of   \textbf {  implicit  }   learning.
\newline  [  95  ]   This is important because previous studies using  distorted visual feedback have suggested that locomotor learning relies  primarily on   \textbf {  implicit  }   learning.
\newline  [  96  ]  2016 Aug 3.  Motor outcomes of feedback delays and   \textbf {  implicit  }  /explicit strategy use:  experimental considerations and clinical implications.  Oswald K(1), Campbell R(2), Wright M(2).  Author information: (1)Department of Psychology, Eastern Michigan University, Ypsilanti, Michigan  koswald1@emich.edu. (2)Department of Psychology, Eastern Michigan University, Ypsilanti, Michigan.  Sensorimotor adaptation requires integration of internal models of motor control  with feedback from the external environment.
\newline  [  97  ]  Apr 27.  Target size matters: target errors contribute to the generalization of   \textbf {  implicit  }    visuomotor learning.  Reichenthal M(1), Avraham G(2), Karniel A(2), Shmuelof L(3).  Author information: (1)Department of Biomedical Engineering, Ben-Gurion University of the Negev,  Beersheva, Israel; Department of Physiology and Cell Biology, Ben-Gurion  University of the Negev, Beersheva, Israel; and. (2)Department of Biomedical Engineering, Ben-Gurion University of the Negev,  Beersheva, Israel; Zlotowski Center for Neuroscience, Ben-Gurion University of  the Negev, Beersheva, Israel. (3)Department of Brain and Cognitive Sciences, Ben-Gurion University of the  Negev, Beersheva, Israel; Department of Physiology and Cell Biology, Ben-Gurion  University of the Negev, Beersheva, Israel; and Zlotowski Center for  Neuroscience, Ben-Gurion University of the Negev, Beersheva, Israel  shmuelof@bgu.ac.il.  The process of sensorimotor adaptation is considered to be driven by errors.  While sensory prediction errors, defined as the difference between the planned  and the actual movement of the cursor, drive implicit learning processes, target  errors (e.g., the distance of the cursor from the target) are thought to drive  explicit learning mechanisms.
\newline  [  98  ]   We hypothesize that in a dynamic reaching environment, where subjects  have to hit moving targets and the targets' dynamic characteristics affect task  success,   \textbf {  implicit  }   processes will benefit from target errors as well.
\newline  [  99  ]   This  result provides evidence that   \textbf {  implicit  }   adaptation is sensitive to target errors.  Copyright © 2016 the American Physiological Society.  DOI: 10.1152/jn.00830.2015 PMCID: PMC4969381
\newline  [  100  ]  Mar 29.  Estimating the   \textbf {  implicit  }   component of visuomotor rotation learning by  constraining movement preparation time.  Leow LA(1), Gunn R(2), Marinovic W(2)(3), Carroll TJ(2).  Author information: (1)Centre for Sensorimotor Performance, School of Human Movement and Nutrition  Sciences, Building 26B, The University of Queensland, Brisbane, Queensland,  Australia; and l.leow@uq.edu.au. (2)Centre for Sensorimotor Performance, School of Human Movement and Nutrition  Sciences, Building 26B, The University of Queensland, Brisbane, Queensland,  Australia; and. (3)School of Psychology and Speech Pathology, Curtin University, Bentley,  Western Australia, Australia.  When sensory feedback is perturbed, accurate movement is restored by a  combination of implicit processes and deliberate reaiming to strategically  compensate for errors.
\newline  [  101  ]   Here, we directly compare two methods used previously to  dissociate   \textbf {  implicit  }   from explicit learning on a trial-by-trial basis: 1) asking  participants to report the direction that they aim their movements, and  contrasting this with the directions of the target and the movement that they  actually produce, and 2) manipulating movement preparation time.
\newline  [  102  ]   The rate and extent of  error reduction under preparation time constraints were similar to estimates of    \textbf {  implicit  }   learning obtained from self-report without time pressure, suggesting  that participants chose not to apply a reaiming strategy to correct visual  errors under time pressure.
\newline  [  103  ]   Surprisingly, participants who reported aiming  directions showed less   \textbf {  implicit  }   learning according to an alternative measure,  obtained during trials performed without visual feedback.
\newline  [  104  ]   This suggests that the  process of reporting can affect the extent or persistence of   \textbf {  implicit  }   learning.  The data extend existing evidence that restricting preparation time can suppress  explicit reaiming and provide an estimate of implicit visuomotor rotation  learning that does not require participants to report their aiming  directions.NEW & NOTEWORTHY During sensorimotor adaptation, implicit  error-driven learning can be isolated from explicit strategy-driven reaiming by  subtracting self-reported aiming directions from movement directions, or by  restricting movement preparation time.
\newline  [  105  ]   The  self-report method produced a discrepancy in   \textbf {  implicit  }   learning estimated by  subtracting aiming directions and implicit learning measured in no-feedback  trials.  Copyright © 2017 the American Physiological Society.  DOI: 10.1152/jn.00834.2016 PMCID: PMC5539449
\newline  [  106  ]   However, improvement was jagged rather  than smooth and performance remained unstable even after 8 days of continually  inverted vision, suggesting that subjects improve via an unknown mechanism,  perhaps a combination of cognitive and   \textbf {  implicit  }   strategies.
\newline  [  107  ]   An explicit strategy prevails when the cerebellum fails to compute movement  errors.  Taylor JA(1), Klemfuss NM, Ivry RB.  Author information: (1)Department of Psychology, University of California, Berkeley, 3210 Tolman  Hall #1650, Berkeley, CA 94720-1650, USA. jordan.a.taylor@berkeley.edu  In sensorimotor adaptation, explicit cognitive strategies are thought to be  unnecessary because the motor system   \textbf {  implicit  }  ly corrects performance throughout  training.
\newline  [  108  ]   This counterintuitive result has been attributed to the independence of    \textbf {  implicit  }   motor processes and explicit cognitive strategies.
\newline  [  109  ]   The cerebellum has  been hypothesized to be critical for the computation of the motor error signals  that are necessary for   \textbf {  implicit  }   adaptation.
\newline  [  110  ]   We explored this hypothesis by  testing patients with cerebellar degeneration on a motor learning task that puts  the explicit and   \textbf {  implicit  }   systems in conflict.
\newline  [  111  ]   These results further reveal  the critical role of the cerebellum in an   \textbf {  implicit  }   adaptive process based on  movement errors and suggest an asymmetrical interaction of implicit and explicit  processes.  DOI: 10.1007/s12311-010-0201-x PMCID: PMC2996538
\end{document}
